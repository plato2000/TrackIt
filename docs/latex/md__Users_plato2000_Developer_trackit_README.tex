\subsubsection*{Written by People Who Code}

{\itshape  Cindy Chen, Parth Oza, Jagan Prem, Kevin Shen }

\subsection*{What is this repository for?}

This repository is for Cindy Chen, Parth Oza, Jagan Prem, and Kevin Shen\textquotesingle{}s 2015-\/2016 I\+DT Automated Software Testing project -\/ automated package tracking and analysis.

\subsection*{How do I get set up?}

\subsubsection*{Summary of set up}

Run {\ttfamily git clone} using the U\+RL on the sidebar to clone the repository.

We are only using the {\ttfamily master} branch for release builds -\/ as in when everything is working, we merge to master and then go back to develop for other things.

To switch to the {\ttfamily master} branch, do {\ttfamily git checkout master}. Before and after every {\ttfamily git} operation you do, you should do {\ttfamily git pull} to prevent merge conflicts.

After we get basic stuff done, we will switch to using different branches for each person.

To add your changes to the {\ttfamily index}, you do {\ttfamily git add -\/A} or {\ttfamily git add F\+I\+L\+E\+N\+A\+ME} if you just want to add a few files.

To commit your changes, run {\ttfamily git commit -\/m \char`\"{}message\char`\"{}}, where message is your commit message.

After you commit, run {\ttfamily git push}, which will send everything to the server.

Oh, and you should also probably have {\ttfamily groovy} installed. That\textquotesingle{}s what the I\+DT event simulator uses.

If you do not have Python 2.\+7.$\ast$ installed, visit \mbox{[}\href{https://www.python.org/downloads/}{\tt https\+://www.\+python.\+org/downloads/}\mbox{]}(Python install site) and download the latest version of Python 2.\+7.

If you do not have My\+S\+QL installed, visit \mbox{[}\href{https://dev.mysql.com/downloads/installer/}{\tt https\+://dev.\+mysql.\+com/downloads/installer/}\mbox{]}(My\+S\+QL install site) and install the correct version of My\+S\+QL for your server / system.

\subsubsection*{Configuration}

The code that is on here is serverside. Our server is going to be separate from the I\+DT server, though it may be on the same machine.

\subsubsection*{Dependencies}

The current dependencies are\+:


\begin{DoxyCode}
1 flask 0.10.1
2 geopy 1.11.0
3 MySQLdb 1.2.5
4 python-dateutil 2.4.1
5 gpxpy 1.0.0
\end{DoxyCode}


If you add libraries to your Python installation, add them to this list so the rest of us know what to get.

Generally, to install Python libraries, you just run {\ttfamily pip2 install L\+I\+B\+R\+A\+R\+Y\+N\+A\+ME}, where {\ttfamily L\+I\+B\+R\+A\+R\+Y\+N\+A\+ME} is the name of the library. {\ttfamily pip} may not be in your {\ttfamily P\+A\+TH}. If this is the case, add your python2 installation to path and run {\ttfamily python -\/m pip install L\+I\+B\+R\+A\+R\+Y\+N\+A\+ME}.

\subsubsection*{Database configuration}

Install My\+S\+QL Installer at \mbox{[}\href{https://dev.mysql.com/downloads/installer/}{\tt https\+://dev.\+mysql.\+com/downloads/installer/}\mbox{]}(My\+S\+QL install site) (mysql-\/installer-\/web-\/community-\/5.\+7.\+10.\+0.\+msi -\/ this is the first option)

Follow the instructions on this link to successfully install My\+S\+QL\+: \mbox{[}\href{http://corlewsolutions.com/articles/article-21-how-to-install-mysql-server-5-6-on-windows-7-development-machine}{\tt http\+://corlewsolutions.\+com/articles/article-\/21-\/how-\/to-\/install-\/mysql-\/server-\/5-\/6-\/on-\/windows-\/7-\/development-\/machine}\mbox{]}(Installation tutorial)

After installation, follow the Post-\/installation Steps on this page to set a root password\+: \mbox{[}\href{http://www.tutorialspoint.com/mysql/mysql-installation.htm}{\tt http\+://www.\+tutorialspoint.\+com/mysql/mysql-\/installation.\+htm}\mbox{]}(Set root password)

\subsubsection*{How to run tests}

We do not have unit tests yet. However, the solution can be tested by running {\ttfamily \hyperlink{package__events_8py}{I\+D\+T\+\_\+\+Server/package\+\_\+events.\+py}} while the server is running to receive test data from the {\ttfamily .gpx} files provided.

\subsection*{How to add stuff to the repo}

\subsubsection*{Contribution guidelines}

Before you commit code, it should be commented properly and well-\/documented. It should also follow P\+E\+P-\/8.

\subsubsection*{Writing tests}

For later

\subsubsection*{Code review}

Have comments, but not too many comments. For Python, follow the P\+E\+P-\/8 conventions. Py\+Charm has plugins that do this for you and alert if you do something wrong. (If you noticed, this document doesn\textquotesingle{}t follow P\+E\+P-\/8, but it doesn\textquotesingle{}t matter since it\textquotesingle{}s not in Python \+:D)

Have a docstring at the beginning of each function/class that documents what it does. Have a comment before the line of obscure bits of code to identify them. DO N\+OT have inline comments.

\subsubsection*{Other guidelines}

Save early, save often. Commit early, commit often. Write good commit messages. Don\textquotesingle{}t wait to comment until the end. 